\chapter{System Study Report}
\section{Module Description}

\subsection*{Prisoner Management Module}
\begin{itemize}
    \item \textbf{Description}: The Prisoner Management module ensures that all details regarding prisoners are accurately recorded and easily accessible. This module allows administrators to add new prisoners, update personal and legal information, and remove prisoners when necessary. It is essential for tracking prisoner status, legal cases, and managing overall prison population data.
    \item \textbf{Input}: Prisoner details (name, ID, crime details, sentence information), Updated prisoner details.
    \item \textbf{Output}: Updated prisoner records, deletion of prisoner records.
\end{itemize}

\subsection*{Visitor Management Module}
\begin{itemize}
    \item \textbf{Description}: The Visitor Management module facilitates the tracking and management of visitors coming into the facility. Administrators can register new visitors, link them to specific prisoners, and remove visitor records when required. This ensures that only authorized individuals can visit prisoners, enhancing both security and accountability within the prison.
    \item \textbf{Input}: Visitor details (name, prisoner ID, visit details), Visitor ID.
    \item \textbf{Output}: Updated visitor records, deletion of visitor records.
\end{itemize}

\subsection*{Cell Management Module}
\begin{itemize}
    \item \textbf{Description}: The Cell Management module is responsible for overseeing the allocation and management of prison cells. It allows administrators to add new cells to the system, delete unused or obsolete cells, and reallocate prisoners between cells to maintain safety, security, and optimal prison operations. This ensures a well-organized facility and efficient use of resources.
    \item \textbf{Input}: Cell details (Cell ID, Cell Location), Prisoner ID, Reallocation details.
    \item \textbf{Output}: Updated cell and prisoner allocation.
\end{itemize}

\subsection*{Staff Management Module}
\begin{itemize}
    \item \textbf{Description}: The Staff Management module enables the efficient handling of prison staff records. Administrators can register new staff members, remove staff when necessary, and manage user access for various staff roles. By ensuring that staff have the appropriate permissions, this module helps maintain secure and effective prison operations.
    \item \textbf{Input}: Staff details (name, role, contact details), User ID.
    \item \textbf{Output}: Updated staff records, user access management.
\end{itemize}

\subsection*{Crime Management Module}
\begin{itemize}
    \item \textbf{Description}: The Crime Management module is used to manage and update the criminal records of prisoners. It allows administrators to add crime details related to each prisoner, update existing crime information, or remove outdated records. This ensures accurate tracking of prisoner histories and supports legal proceedings within the correctional facility.
    \item \textbf{Input}: Crime details (crime type, date, prisoner ID), Updated crime information.
    \item \textbf{Output}: Updated crime records, removal of crime details.
\end{itemize}

\subsection*{Job Management Module}
\begin{itemize}
    \item \textbf{Description}: The Job Management module allows for the efficient assignment and management of tasks given to prisoners. Administrators can add job details, update existing tasks, and remove completed or redundant job assignments. This module helps track the work done by prisoners, contributing to their rehabilitation and prison maintenance efforts.
    \item \textbf{Input}: Job details (task description, prisoner ID, job assignment), Updated job details.
    \item \textbf{Output}: Updated prisoner job assignments, removal of job records.
\end{itemize}

\subsection*{Work Management Module}
\begin{itemize}
    \item \textbf{Description}: The Work Management module tracks the work done by prisoners within the facility. It allows administrators to record details of tasks or jobs completed by prisoners, including work hours, and ensures proper documentation of all work activities. The module also enables the calculation of total work hours logged by each prisoner for monitoring productivity and compliance.
    \item \textbf{Input}: Work details (task completed, prisoner ID, work hours), Updated work details.
    \item \textbf{Output}: Updated work records, total work hours for prisoners.
\end{itemize}